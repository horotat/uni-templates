
\renewcommand{\abstractname}{Zusammenfassung}    % clear the title
%\renewcommand{\absnamepos}{empty}

\begin{abstract}


Roboter erfahren in der Neurowissenschaft erh\"ohte Aufmerksamkeit als
Mittel zur Verifizierung theoretischer Modelle der Enttwicklung
sozialer F\"ahigkeiten. Insbesondere humanoide Roboter in der Gr\"osse
kleiner Kinder eignen sich als kontrollierte Platform zur Simulation
von Interaktionen zwischen Erwachsenen und Kleinkinder.
Solche Roboter in Interaktion mit Menschen, ausgestattet mit biologisch inspirierten Modellen der
Entwicklung sozialer und kognitiver F\"ahigkeiten, k\"onnten wertwolle
Einblicke in die Lernmechanismen gew\"ahren, die Kleinkinder bei der
sozialen Interaktion nutzen.
Ein solcher Mechanismus, der sich in der fr\"uhen Kindheit entwickelt,
ist die F\"ahigkeit, Aufmerksamkeit mit anderen zu teilen.
Bei Kleinkindern basiert das Verhalten beim Zeigen auf
gemeinsamer Aufmerksamkeit und tritt erst nach dem Erwerb
zugrundeliegender Hand-Auge-Koordination auf.
In dieser Masterarbeit versuchen wir zu erkl\"aren, wie sich Zeigen aus
sensorimotorischem Lernen der Hand-Auge-Koordination entwickelt,
am Beispiel eines humanoiden Roboters. Mittels Motorplappern (engl. 
motor babbling) erlernt der Roboter Gelenkkonfigurationen f\"ur verschiedene 
Armpositionen.
Die motorabh\"angigen Gelenkskonfigurationen wurden in einem
Babbling-Experiment erfasst und dann zum Training interner, biologisch
inspirierter und auf selbstorganisierenden Karten
beruhender Modelle benutzt.
Wir trainieren und analysieren Modelle mit verschiedenen
Gr\"ossen und vergleichen diese, basierend auf ihrer
Konfiguration, mit unterschiedlichen Entwicklungsstufen
sensorimotorischer F\"ahigkeit.
Schlussendlich zeigen wir anhand der Implementierung eines solches
Modells auf einem Roboter eine M\"oglichkeit auf, die Entwicklung des
Zeigens aus der Hand-Auge-Koordination zu erkl\"aren. Dem Roboter werden
in einem Experiment Objekte ausserhalb seiner Reichweite pr\"asentiert,
worauf der Roboter mit einer Zeigegeste reagiert.

\bigskip
\end{abstract}